\documentclass{ltjsarticle}
\usepackage{graphicx}
\begin{document}
\begin{center}
    \Large{感染症とその予防}
\end{center}
\begin{flushright}
    2年2組 22番 ○○
\end{flushright}

\section{はじめに}

5年前、世界を恐怖のどん底に陥らせた原因不明の感染症。現在は「新型コロナ」と呼ばれているが、当時は世界中で恐れられ、3月には「全国一斉休校」という前代未聞の措置が講じられたのは記憶に新しい。クルーズ船「ダイヤモンド・プリンセス」号のあたりからだろうか、人々の感染症対策の意識が急激に高まっていったが、全く抑えこむことはできず、日本国内でも大流行した。当時は「3密」や「新しい生活様式」、「リモートワーク」など、今までになかった新しい概念を導入したりと、政府は必死で対応にあたっていた。

しかし、現在では、「withコロナ」などと言って、従来のような生活を送れるようになっていった。治療法も確立され、現在ではインフルエンザと同等、またはそれ以下として扱われるような存在となった。むしろ「マイコプラズマ」のほうが恐れられているようにまで感じる。本レポートでは、感染症の予防について、感染経路の遮断という視点で考察するとともに、人々の意識の変化について考察する。

\section{感染症とは}

とはいえ、まずは感染症を定義しなければならない。

本レポートにおいて、感染症とは、病原体が呼吸器より体内に侵入し、体内で増殖することにより発症するものとする。病原体が細菌かウイルスかは問わない。\\
そのため、性的接触が主な感染経路である性感染症や、傷口に病原体が入ることで感染する「破傷風」「劇症型溶血性レンサ球菌感染症」\footnote[1]{通称「人食いバクテリア」。傷口などから体内に入り、その部分の組織の壊死などを引きおこす。そうなった場合は緊急手術で切断する以外の選択肢はない。また、この手術が遅れたり、呼吸器から感染したりすると、白血球数が減少し、場合によっては手足や内臓が壊死し死に至る。厳密には感染組織の壊死は「壊死性軟部組織感染症」というが、この状態になると40〜50\%程度の確立で「劇症型溶血性レンサ球菌感染症」を引き起こす。本来まれな病気であるが、発症すると致死率は約30\%とかなり高い水準の恐ろしい感染症である。なお、致死率30\%の感染症は他にクリミア・コンゴ出血熱(1類感染症)などがある。}などは除外する。

\section{感染症の感染経路}

感染症の感染経路には主に以下の4種類がある。

\subsection{飛沫感染}

感染者の咳やくしゃみなどの飛沫を吸入することで感染すること。約1mの範囲で感染力をもつ。\\
風邪やインフルエンザなど、多くの感染症はこの経路で感染する。

\subsection{空気感染}

病原体が空気中に飛びだし、漂っている状態のものを吸入することで感染すること。1mを超えた範囲でも感染する。\\
はしか、水ぼうそう、結核などが代表的。

\subsection{接触感染}

感染者と皮膚や粘膜の直接接触、または物体の表面を介しての間接接触により感染すること。本レポートにおいては考察しない。\\
性感染症、破傷風、プール熱などが代表的。

\subsection{経口感染}

病原体に汚染された食品を、十分な処理をしない状態や調理過程で付着した状態などで喫食して感染すること。糞便が手指を介して経口摂取される場合を特に糞口感染という。\\
食中毒、感染性胃腸炎\footnote[2]{ノロウイルスやロタウイルスなど}などが代表的。

\section{代表的な感染症対策の考察}

ここでは、新型コロナウイルスが流行してたときに実施していた感染症対策について考察する。

\subsection{手洗い・うがい}

これの主な目的は、手や喉に付着した病原体を洗いながし、病原体の侵入を防止することである。これは、すべての感染経路に対して有効な対策であり、調理前の手洗いは特に経口感染の防止に効果があると考えられる。ただし、手洗いは正しい方法でやらなければあまり意味はない。

\subsection{マスク}

これの主な目的は、吸入する空気にフィルターを設けることにより、病原体が喉に侵入しないようにすることである。これについても、飛沫感染および空気感染による感染症には効果があると考えられる。しかし、当然ながらこれも、適切にマスクを着用しなければあまり意味はない。\\
また、コロナ渦の影響で周りがマスクをしていることにより、花粉症の人が花粉症対策としてマスクを着用することに対する抵抗感がかなり薄くなったのはありがたかった。

\section{潜伏期間と感染力}

現在では、新型コロナもかなり落ちついて、マスクなしでの生活がスタンダードになりつつあるが、周りと話していると「感染症は症状があるときのみ感染力をもつもの」であるという誤認識がみられたため、主要な感染症の潜伏期間、および治療を調査した。\\
もちろん、潜伏期間中も人にうつす可能性があることに注意されたい。

\subsection{新型コロナ・インフルエンザ}

潜伏期間は5日〜2週間。発症後5日間は出席停止になる。\\
主に飛沫感染により広がり、この5日間は他人に感染させるリスクが高くなる。\\
インフルの場合、治療は従来型の「タミフル」を1週間程度服用、または新型の「ゾフナール」を1回服用。
コロナの場合は対処療法のことが多い。

\subsection{マイコプラズマ}

感染力が強く、重症化すると肺炎を引きおこしたりする。\\
潜伏期間は2〜3週間。特徴的な咳は解熱後3週間程度続くことが多い。\\
治療は抗菌薬(抗生物質)を服用。

\subsection{溶連菌感染症}

潜伏期間は2〜3日。病原体自体は例の人食いバクテリアと同じ。\\
高熱が特徴で急性期は感染力が強いものの、抗生物質の服用後24時間程度で感染力は失われる。\\
治療は抗菌薬(抗生物質)を服用。

\section{おわりに}

今回は、主な感染症として、新型コロナとインフルエンザ、マイコプラズマと溶連菌について特に取りあげたが、感染症対策は他の感染症にも共通して言えることであることに注意されたい。かくいう私もマイコプラズマに感染したりしているので、きちんと感染症対策はしなければならないと感じた。かかりつけの医師いわく、コロナ渦で感染症対策を徹底していた影響で、子どもの免疫が下がって子どもの感染症が増えているらしい。感染症対策でもやりすぎはよくないようだ。

\end{document}
